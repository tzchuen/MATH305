\documentclass[a4paper, titlepage, DIV=14]{scrartcl}
\usepackage{import}
\usepackage{xifthen}
\usepackage{pdfpages}
\usepackage{transparent}
\usepackage{amsmath,amssymb,amsthm}
\usepackage{bm}
\usepackage{biblatex}
\usepackage{enumerate}
\usepackage{caption}
\usepackage{float}
\usepackage[colorlinks=true, allcolors=black, urlcolor=cyan]{hyperref}
\usepackage{graphicx}
\usepackage[framed,numbered]{matlab-prettifier}
\usepackage{mathtools}
%\usepackage{mcode}
\usepackage[headsepline]{scrlayer-scrpage}
\usepackage{graphicx}
\usepackage{physics}
\usepackage{siunitx}
\usepackage{cancel}
\usepackage{appendix}
\usepackage{lipsum}
\usepackage{listings}

\newcommand{\incfig}[1]{%
    \def\svgwidth{0.8\columnwidth}
    \import{./}{#1.pdf_tex}
}

\title{Assignment 2}
\subtitle{MATH 305 - Applied Complex Analysis}
\author{Zhi Chuen Tan (65408361)}
\date{2020W2}
\publishers{
    \includegraphics[width=0.65\textwidth]{mathlogo.eps}}

\setkomafont{pageheadfoot}{%
\normalcolor}

\newcommand{\Arg}{\text{Arg}}


\ohead{MATH 305 201 \\ 2020W2}
\chead{\Large{Assignment 2}}
\ihead{Zhi Chuen Tan \\ 65408361}

\usepackage{setspace} %For line spacing

\setlength{\parindent}{0pt} % Disable indentation

\begin{document}
    \onehalfspacing
    \hypersetup{pageanchor=false}
    \begin{titlepage}
        \maketitle
        \vfill
        
    \end{titlepage}
    \hypersetup{pageanchor=true}

    \begin{enumerate}
        \item For this problem, let $z=x+yi$, $w=u+vi$.
        
        \begin{enumerate}[i)]
            \item LHS:
            \begin{align*}
                \overline{z+w} &= \overline{(x+yi)+(u+vi)} \\
                            &= \overline{(x+u) + (y+v)i} \\
                            &= (x+u) - (y+v)i
            \end{align*}
            RHS:
            \begin{align*}
                \bar{z} + \bar{w} &= (x-yi) + (u-vi)i \\
                            &= (x+u) + (-y-v)i \\
                            &= (x+u) - (y+v)i \\
                            &= \text{LHS}
            \end{align*} 
            \qed
            
            \item LHS:
            \begin{align*}
                \overline{zw} &= \overline{(x+yi)(u+vi)} \\
                &= \overline{(xu-yv) + (xv+yu)i} \\
                &= (xu-yv) - (xv+yu)i
            \end{align*}
            RHS:
            \begin{align*}
                \bar{z}\bar{w} &= (x-yi)(u-vi) \\
                    &= (xu - yv) + (-xv-yu)i \\
                    &= (xu - yv) - (xy + yu)i \\
                    &= \text{LHS}
            \end{align*}\qed
            
            \item
            LHS:
            \begin{align*}
                |\bar{z}| &= |(x-yi)| \\
                        &= \sqrt{x^{2}+(-y)^{2}} \\
                        &= \sqrt{x^{2}+y^{2}}
            \end{align*}
            RHS:
            \begin{align*}
                |z| &= |x+yi| \\
                    &= \sqrt{x^{2}+y^{2}} \\
                    &= \text{LHS} 
            \end{align*}\qed

            \item From Problem 1.3 above, we have seen that:
            \begin{equation*}
                |z| = \sqrt{x^{2}+y^{2}}
            \end{equation*}
            \begin{align*}
                 |\Re(z)| &= |x| \\
                \therefore  |x| &\leq \sqrt{x^{2}+y^{2}} \\
                \Rightarrow |\Re(z)| &\leq |z| \\
                |\Im(z)| &= |y| \\
                \therefore |y| &\leq \sqrt{x^{2}+y^{2}} \\
                \Rightarrow |\Im(z)| &\leq |z|
            \end{align*}\qed

            \item We have that:
            \begin{align*}
                z\bar{w} &= (x+yi)(u-vi) \\
                    &= (xu+yv) + (-xv+yu)i
            \end{align*}
            
            LHS:
            \begin{align*}
                |z+w|^{2} &= (z+w)\overline{z+w} \\
                    &= (z+w)(\bar{z}+\bar{w}), \text{ as proven in Problem 1.1} \\
                    &= z\bar{z} + z\bar{w} + w\bar{z} + w\bar{w} \\
                    &= |z|^{2} + z\bar{w} + w\bar{z} + |w|^{2} \\
                    &= |z|^{2} + |w|^{2} + (x+yi)(u-vi) + (u+vi)(x-yi) \\
                    &= |z|^{2} + |w|^{2} + (xu+yv)+(-xv+yu)i + (ux + yv) + (-yu+xv)i \\
                    &= |z|^{2} + |w|^{2} + (xu+yv)+\cancel{(yu-xv)i} + (xu+yv) \cancel{-(yu-xv)i} \\
                    &= |z|^{2} + |w|^{2} + 2(xu+yv) \\
                    &= |z|^{2} + |w|^{2} + 2\Re(z\bar{w}), \text{ using the result above} \\
                    &= \text{RHS}
            \end{align*}\qed
            
            \item From Problem 1.5, we have:
            \begin{align*}
                |z + w|^{2} &= |z|^{2} + |w|^{2} + 2\Re(z\bar{w}) \\
                |z + w|^{2} &\leq |z|^{2} + |w|^{2} + 2|z\bar{w}|, \text{ from Problem 1.4}\\
                |z + w|^{2} &\leq |z|^{2} +|w|^{2} + 2|z||\bar{w}|, \text{ from Problem 1.2} \\
                |z + w|^{2} &\leq |z|^{2} +|w|^{2} + 2|z||w|, \text{ from Problem 1.3} \\
                |z + w|^{2} &\leq (|z| + |w|)^{2}, \text{ from }(a+b)^{2} = (a^{2}+2ab+b^{2}) \\
                |z + w| &\leq |z| + |w|, \text{ after taking the square root of both sides}
            \end{align*} \qed
            
            \item We have that:
            \begin{align}
                |-z| &= |(-x-yi)| \nonumber \\    
                    &= \sqrt{(-x)^{2}+(-y)^{2}} \nonumber \\ 
                    &= \sqrt{x^{2}+y^{2}} \nonumber \\ 
                    &= |z|, \label{eq:-z}
            \end{align} and that the AM-GM inequality is:
            \begin{equation}
                \frac{a+b}{2} \geq \sqrt{ab}, \, \{a, b: a\in \mathbb{R}^{+}, \, b\in \mathbb{R}^{+}\}
            \end{equation}
            
            With the above, and the triangle inequality from Problem 1.6, we have:
            \begin{align*}
                |z-w| &\geq \biggr||z|-|w|\biggr| \\
                |z-w|^{2} &\geq (|z|-|w|)^{2} \\
                |z-w|^{2} &\geq |z|^{2} - 2|z||w| + |w|^{2} \\
                |z|^{2} + |-w|^{2} + 2\Re(z(-\overline{w})) 
                    &\geq |z|^{2} - 2|z||w| + |w|^{2}, \text{ from Problem 1.5}\\
                \cancel{|z|^{2}} + \cancel{|w|^{2}} + 2\Re((x+yi)(-(u-vi))) 
                    &\geq \cancel{|z|^{2}} - 2|z||w| + \cancel{|w|^{2}}, \text{ from \autoref{eq:-z}} \\
                \cancel{2}\Re((x+yi)(-(u-vi))) &\geq - \cancel{2}|z||w| \\
                -(xu+yv) &\geq -\sqrt{x^{2}+y^{2}}\sqrt{u^{2}+v^{2}} \\
                (xu+yv) &\leq \sqrt{x^{2}+y^{2}}\sqrt{u^{2}+v^{2}} \\
                (xu+yv)^{2} &\leq (x^{2}+y^{2})(u^{2}+v^{2}) \\
                \cancel{(xu)^{2}} + 2xuyv + \cancel{(yv)^{2}} &\leq \cancel{(xu)^{2}} 
                    + (xv)^{2} + (yu)^{2} + \cancel{(yv)^{2}} \\
                2xuyv &\leq (xv)^{2} + (yu)^{2} \\
                \frac{(xv)^{2}+(yu)^{2}}{2} &\geq xvyu \\
                \frac{(xv)^{2}+(yu)^{2}}{2} &\geq \sqrt{(xv)^{2}(yu)^{2}} \\
                \equiv \frac{a+b}{2} &\geq \sqrt{ab}, 
            \end{align*}where $a=(xv)^{2}, \, b=(yu)^{2}$. As the inequality reduces to the
            AM-GM inequality (which is known), the inequality $|z-w|\geq||z|-|w||$ 
            is valid. \qed \\
        \end{enumerate}

        \item 
        \begin{enumerate}[i)]
            \item We have that $(a+b)^{3} = a^{3} + b^{3} +3ab(a+b)$:
            \begin{align*}
                \sin(3\theta) &= \Im(e^{i3\theta}) \\
                    &= \Im((e^{i\theta})^{3}) \\
                    &= \Im(\cos^{3}(\theta)+i^{3}(\sin^{3}(\theta)) 
                        + 3\cos(\theta)(i\sin(\theta))(\cos(\theta)+i\sin(\theta))) \\
                    &= \Im(cos^{3}(\theta)+ii^{2}(\sin^{3}(\theta)) 
                        + 3i\cos(\theta)\sin(\theta)(\cos(\theta)+i\sin(\theta))) \\
                    &= \Im(\cos^{3}(\theta)-i(\sin^{3}(\theta)) 
                        + 3i\cos^{2}(\theta)\sin(\theta) - 3\cos(\theta)\sin(\theta))) \\
                    &= \Im((\cos^{3}(\theta)-3\cos(\theta)\sin(\theta))+(3\cos^{2}(\theta)\sin(\theta)-\sin^{3}(\theta))i) \\
                    &= 3\cos^{2}(\theta)\sin(\theta)-\sin^{3}(\theta)
            \end{align*} \qed

            \item 
            \begin{align*}
                \sin(\theta - \psi) &= \Im(e^{i(\theta-\psi)}) \\
                    &= \Im(e^{i\theta}e^{-i\psi}) \\
                    &= \Im([\cos(\theta)+i\sin(\theta)][\cos(\psi)-i\sin(\psi)]) \\
                    &= \Im(\cos(\theta)\cos(\psi) - i\cos(\theta)\sin(\psi) 
                        + i\sin(\theta)\cos(\psi) + \sin(\theta)\sin(\psi)) \\
                    &= \sin(\theta)\cos(\psi)-\cos(\theta)\sin(\psi)
            \end{align*} \qed
        \end{enumerate} 

        \item All sketches in this problem are vector (\verb_.svg_) files, and can be zoomed in without loss
        in quality.
        \begin{enumerate}[i)]
            \item The sketch for the domain $\Omega \subset \mathbb{C}$ is shown in \autoref{fig:3.1.a}.
            \begin{figure}[!h]
                \centering
                \incfig{3ai}
                \caption{}
                \label{fig:3.1.a}
            \end{figure}
            
            We can observe that the mapping $f$ is a composite of three functions, i.e.
            $f(z) = (f_{3} \circ f_{2} \circ f_{1})(z)$, where:
            \begin{align*}
                f_{1}(z) &= e^{i\frac{\pi}{2}} z, \text{ which represents a rotation of } \frac{\pi}{2}, \\
                f_{2}(w) &= 2w, \text{ which represents a dilation of the set by factor 2,} \\
                f_{3}(s) &= s + (2+2i), \text{ which represents a translation by }(2+2i),
            \end{align*}which can be verified by finding $f_{3}(f_{2}(f_{1}(z)))$:
            \begin{align*}
                f_{3}(f_{2}(f_{1}(z))) &= f_{3}(f_{2}(e^{i\frac{\pi}{2}})) \\
                        &= f_{3}(2e^{i\frac{\pi}{2}}) \\
                        &= 2e^{i\frac{\pi}{2}} + (2+2i) \\
                        &= f(z) 
            \end{align*} \qed \\
            Therfore, the image of the domain $\Omega$, $f(\Omega)$, is shown in \autoref{fig:3.1.b} \\
            \begin{figure}[!h]
                \centering
                \incfig{3a2}
                \caption{}
                \label{fig:3.1.b}
            \end{figure} 

            \item
            The sketch for the domain $\Omega \subset \mathbb{C}$ is shown in \autoref{fig:3.2.a}. 
            We define the set $\{\zeta \in \mathbb{C} \, | \, \zeta = e^{\frac{\pi}{2}z}, \text{ for } -1<\Im(z)<1\}$:
            \begin{figure}[!h]
                \centering
                \incfig{3b1}
                \caption{}
                \label{fig:3.2.a}
            \end{figure}
            \begin{align*}
                \zeta &= e^{\frac{\pi}{2}z} \\
                    &= (e^{z})^{\frac{\pi}{2}} \\
                    &= (e^{\Re(z)+\Im(z)i})^{\frac{\pi}{2}} \\
                    &= (e^{\Re(z)} \, e^{\Im(z)i})^{\frac{\pi}{2}} \\
                    &= e^{\frac{\pi}{2}\Re(z)}(e^{\frac{\pi}{2}i\Im(z)}) \\
                    &= e^{\frac{\pi}{2}\Re(z)}(\cos(\frac{\pi}{2}\Im(z))
                        + i\sin(\frac{\pi}{2}\Im(z))) \\
                    &= e^{\frac{\pi}{2}\Re(z)}\cos(\frac{\pi}{2}\Im(z))
                        + ie^{\frac{\pi}{2}\Re(z)}\sin(\frac{\pi}{2}\Im(z)) \\
                \Rightarrow \Im(\zeta) &= e^{\frac{\pi}{2}\Re(z)}\sin(\frac{\pi}{2}\Im(z)) \\
                \sin(\frac{\pi}{2}\Im(z)) &= e^{-\frac{\pi}{2}\Re(z)}\Im(\zeta) \\
                \therefore \Im(z) &= \frac{2}{\pi} \arcsin(e^{-\frac{\pi}{2}\Re(z)}\Im(\zeta))
            \end{align*}
            Now, we have:
            \begin{gather*}
                -1 < \Im(z) < 1 \Leftrightarrow -1 < \frac{2}{\pi} 
                    \arcsin(e^{-\frac{\pi}{2}\Re(z)}\Im(\zeta)) < 1 \\
                -\frac{\pi}{2} < \arcsin(e^{-\frac{\pi}{2}\Re(z)}\Im(\zeta)) < \frac{\pi}{2} \\
                -\sin(\frac{\pi}{2}) < e^{-\frac{\pi}{2}\Re(z)}\Im(\zeta) < \sin(\frac{\pi}{2}) \\
                -e^{\frac{\pi}{2}\Re(z)} < \Im(\zeta) < e^{\frac{\pi}{2}\Re(z)},
            \end{gather*} which gives the image of $f$ as:
            \begin{equation*}
                f(\Omega) = \{\zeta \in \mathbb{C} \, | \, -e^{\frac{\pi}{2}\Re(z)} < \Im(\zeta) < e^{\frac{\pi}{2}\Re(z)}\},
            \end{equation*} which has the sketch as shown in \autoref{fig:3.2.b}.
            \begin{figure}[!h]
                \centering
                \import{./}{3b2.pdf_tex}
                \caption{}
                \label{fig:3.2.b}
            \end{figure}
            
            
            \item The sketch for the domain $\Omega \subset \mathbb{C}$ is shown in \autoref{fig:3.3.a}. We define a set 
            $\{\zeta \in \mathbb{C} \, | \, \zeta = \frac{z-1}{z+1}, \, \Re(z) > 0 \}$. Let $\Re(z)=x, \, \Im(z)=y, \, 
            \Re(\zeta) = u, \, \Im(\zeta) = v$:
            \begin{figure}[!h]
                \centering
                \incfig{3c1}
                \caption{}
                \label{fig:3.3.a}
            \end{figure}
            \begin{align*}
                \zeta &= \frac{z-1}{z+1} \\
                z &= \frac{1+\zeta}{\zeta-1} \\
                    &= \frac{(1+u)+vi}{(u-1)+vi} \\
                    &= ((1+u)+vi)((u-1)+vi)^{-1} \\
                    &= ((1+u)+vi)(\frac{u-1}{(u-1)^{2}+v^{2}} 
                        - i\frac{v}{(u-1)^{2}+v^{2}}) \\
                    &= \biggr(\frac{(u+1)(u-1)}{(u-1)^{2}+v^{2}} 
                        + \frac{v^{2}}{(u-1)^{2}+v^{2}}\biggr) + \ldots \\
                \therefore \Re(z) &= \biggr(\frac{(u+1)(u-1)}{(u-1)^{2}+v^{2}} 
                + \frac{v^{2}}{(u-1)^{2}+v^{2}}\biggr),
            \end{align*}which now gives us:
            \begin{gather*}
                \Re(z) > 0 \Leftrightarrow \biggr(\frac{(u+1)(u-1)}{(u-1)^{2}+v^{2}} 
                + \frac{v^{2}}{(u-1)^{2}+v^{2}}\biggr) > 0 \\
                \frac{(u+1)(u-1)}{\cancel{(u-1)^{2}+v^{2}}} > - \frac{v^{2}}{\cancel{(u-1)^{2}+v^{2}}} \\
                (u+1)(u-1) > -v^{2} \\
                u^{2}-1 > -v^{2} \\
                u^{2} + v^{2} > 1 \\
                \sqrt{u^{2} + v^{2}} > 1 \\
                \Rightarrow |\zeta| > 1,
            \end{gather*} as $|\zeta| = \sqrt{u^{2} + v^{2}}$, which gives the image:
            \begin{equation*}
                f(\Omega) = \{\zeta \in \mathbb{C} \, | \, |\zeta| > 1\},
            \end{equation*} which has a sketch as shown in \autoref{fig:3.3.b}.
            \begin{figure}[!h]
                \centering
                \incfig{3c2}
                \caption{}
                \label{fig:3.3.b}
            \end{figure} \\
        
        \end{enumerate} 
        
        \item 
        \begin{enumerate}[i)]
            \item Since $f \in H(\Omega)$, the Cauchy-Riemann equations:
            \[
            \begin{cases}
                \partial_{x} \, u(x,y) = \partial_{y} \, v(x,y), \\
                \partial_{x} \, v(x,y) = -\partial_{y} \, u(x,y),
            \end{cases} 
            \] must be satisfied. As $f$ is real valued, it will be of the form:
            \begin{equation*}
                f = u(x,y) + iv(x,y),
            \end{equation*} where $v(x,y)=0$. Therefore, we have that:
            \begin{align*}
                \partial_{x} \, v(x,y) &= 0, \\
                \partial_{y} \, v(x,y) &= 0,
            \end{align*} and since the Cauchy-Riemann equations must be satisfied, we 
            must also assert that:
            \begin{align*}
                \partial_{x} \, u(x,y) &= \partial_{y} \, v(x,y) = 0 \\
                \partial_{y} \, u(x,y) &= - \partial_{x} \, v(x,y) = 0
            \end{align*}
            As both $\partial_{x} \, u(x,y)$ and $\partial_{y} \, u(x,y)$ are zero, we must
            have that $u(x,y)$ is constant with respect to both $x$ and $y$, and since $f$ is 
            real-valued as well, we can conclude that:
            \begin{equation*}
                f = u(x,y), \text{ where } u(x,y) = a, \, \{a \in \mathbb{R}\}, \, \text{for } f \in \Omega
            \end{equation*} \qed \\

            
            \item Let $f=u+vi$, $\bar{f} = u-vi$. As both $f, \bar{f} \in H(\Omega)$, they must both
            satisfy the Cauchy-Riemann equations. For $\bar{f}$, we have:
            \begin{align*}
                \partial_{x} \, u &= -\partial_{y} \, v \\
                -\partial_{x} \, v &= -\partial_{y} \, u \\
            \end{align*}
            Similarly, for $f$ we have:
            \begin{align*}
                \partial_{x} \, u &= \partial_{y} \, v \\
                \partial_{x} \, v &= -\partial_{y} \, u \\
            \end{align*}
            As both these sets of equations must be true, we can observe that:
            \begin{equation*}
                \partial_{x} \, u = \partial_{y} \, v = -\partial_{y} \, v = 0,
            \end{equation*} as only $0$ can satisfy the equation $a = -a$. Similarly, we can also obtain that
            \begin{equation*}
                \partial_{x} \, v = \partial_{y} \, u = -\partial_{y} \, u = 0
            \end{equation*}

            As $\partial_{x} \, u = \partial_{x} \, v = \partial_{y} \, u = \partial_{y} \, v = 0$, we can conclude
            that $u(x,y)$ and $v(x,y)$ are independent of $x,y$, and that $f$ is therefore constant on $\Omega$. \qed \\
               

        \end{enumerate}

        \item The Cauchy-Riemann equations are:
        \[
          \begin{cases}
              \partial_{x} \, u(x,y) = \partial_{y} \, v(x,y), \\
              \partial_{x} \, v(x,y) = -\partial_{y} \, u(x,y),
          \end{cases}  
        \] where $z=x+yi, \, f(z) = u + vi$. If the Cauchy-Riemann equations are satisfied, we have that:
        \begin{equation*}
            f'(z) = \partial_{x} \, u(x,y) +i\partial_{x} \, v(x,y) = \partial_{y} \, v(x,y) -i\partial_{y} \, u(x,y)
        \end{equation*} \\

        \begin{enumerate}[i)]
            \item We have that $u(x,y) = e^{-2xy}\cos(x^{2}-y^{2}), \, v(x,y) = e^{-2xy}\sin(x^{2}-y^{2})$:
            \begin{align*}
                \therefore \partial_{x} \, u(x,y) &= -2ye^{-2xy}\cos(x^{2}-y^{2})
                                                - 2xe^{-2xy}(\sin(x^{2}-y^{2}) \\
                \partial_{y} \, v(x,y) &= -2xe^{-2xy}\sin(x^{2}-y^{2}) + (-2y)e^{-2xy}\cos(x^{2}-y^{2}) \\
                                    &= -2ye^{-2xy}\cos(x^{2}-y^{2}) - 2xe^{-2xy}\sin(x^{2}-y^{2}) \\
                \Rightarrow \partial_{y} \, v(x,y) &= \partial_{x} \, u(x,y), \text{ which satisfies the first Cauchy-Riemann equation}
            \end{align*}
            \begin{align*}
                \partial_{x} \, v(x,y) &= -2ye^{-2xy}\sin(x^{2}-y^{2}) + 2xe^{-2xy}\cos(x^{2}-y^{2}) \\
                \partial_{y} \, u(x,y) &= -2xe^{-2xy}\cos(x^{2}-y^{2}) + 2ye^{-2xy}\sin(x^{2}-y^{2}) \\
                                    &= -(2xe^{-2xy}\cos(x^{2}-y^{2}) - 2ye^{-2xy}\sin(x^{2}-y^{2})) \\
                                    &= -\partial_{x} \, v(x,y), \text{ which satisfies the second Cauchy-Riemann equation}
            \end{align*}
            As both Cauchy-Riemann equations are satisfied $\forall x, \, y$, the function: 
            \begin{equation*}
                f(x+iy)=e^{-2xy}(\cos(x^{2}-y^{2})+i\sin(x^{2}-y^{2}))
            \end{equation*}
            is entire. \qed

            Its derivative is therefore:
            \begin{align*}
                f'(z) &= \partial_{x} \, u(x,y) +i\partial_{x} \, v(x,y) \\
                    &= [-2ye^{-2xy}\cos(x^{2}-y^{2})
                        - 2xe^{-2xy}(\sin(x^{2}-y^{2})] \\
                        &\, + i[-2ye^{-2xy}\sin(x^{2}-y^{2}) 
                        + 2xe^{-2xy}\cos(x^{2}-y^{2})] \\
                    &= -2e^{-2xy}(y\cos(x^{2}-y^{2})+x\sin(x^{2}-y^{2})) + i2e^{-2xy}(x\cos(x^{2}-y^{2})-y\sin(x^{2}-y^{2})) \\
            \end{align*}

            \item We have that:
            \begin{align*}
                g(z) &= \frac{1}{2}(e^{iz}+e^{-iz}) \\
                    &= \frac{1}{2}(e^{ix+i^{2}y}+e^{-(ix+i^{2}y)}) \\
                    &= \frac{1}{2}(e^{ix-y} + e^{-ix+y} ) \\
                    &= \frac{1}{2}(e^{ix}e^{-y} + e^{-ix}e^{y}) \\
                    &= \frac{1}{2}( e^{-y}(\cos(x)+i\sin(x)) + e^{y}(\cos(x)-i\sin(x)) ) \\
                    &= \frac{1}{2}(e^{-y}+e^{y})\cos(x) + \frac{1}{2}i(e^{-y}-e^{y})\sin(x), \\
            \end{align*} which gives:
            \begin{equation*}
                u = \frac{1}{2}(e^{-y}+e^{y})\cos(x), \, v=\frac{1}{2}(e^{-y}-e^{y})\sin(x)
            \end{equation*}

            Now, we calculate:
            \begin{align*}
               \partial_{x} \, u &= -\frac{1}{2}(e^{-y}+e^{y})\sin(x) \\
               \partial_{y} \, v &= \frac{1}{2}(-e^{-y}-e^{y})\sin(x) \\
                                &= -\frac{1}{2}(e^{-y}+e^{y})\sin(x) \\
                                &= \partial_{x} \, u \\
                                &\Rightarrow \text{ the first Cauchy-Riemann equation is satisfied} \\
                \partial_{x} \, v &= \frac{1}{2}(e^{-y}-e^{y})\cos(x) \\
                \partial_{y} \, u &= \frac{1}{2}(-e^{-y} + e^{y})\cos(x) \\
                                &= -\frac{1}{2}(e^{-y} - e^{y})\cos(x) \\
                                &= -\partial_{x} \, v \\
                                &\Rightarrow \text{ the second Cauchy-Riemann equation is satisfied} \\
                                &\Rightarrow \text{ the function }g(z)\text{ is entire}
            \end{align*} \qed \\
            We can therefore compute its derivative:
            \begin{align*}
                g'(z) &= \partial_{x} \, u + i\partial_{x} \, v \\
                    &= -\frac{1}{2}(e^{-y}+e^{y})\sin(x) + \frac{1}{2}i(e^{-y}-e^{y})\cos(x)
            \end{align*}
        \end{enumerate}
    \end{enumerate}
\end{document}
\documentclass[a4paper, titlepage, DIV=14]{scrartcl}
\usepackage{import}
\usepackage{xifthen}
\usepackage{pdfpages}
\usepackage{transparent}
\usepackage{amsmath,amssymb,amsthm}
\usepackage{bm}
\usepackage{biblatex}
\usepackage{enumerate}
\usepackage{caption}
\usepackage{float}
\usepackage[colorlinks=true, allcolors=black, urlcolor=cyan]{hyperref}
\usepackage{graphicx}
\usepackage[framed,numbered]{matlab-prettifier}
\usepackage{mathtools}
%\usepackage{mcode}
\usepackage[headsepline]{scrlayer-scrpage}
\usepackage{graphicx}
\usepackage{physics}
\usepackage{siunitx}
\usepackage{cancel}
\usepackage{appendix}
\usepackage{lipsum}
\usepackage{listings}
\usepackage{fontawesome}

\newcommand{\incfig}[1]{%
    \def\svgwidth{0.75\columnwidth}
    \import{./}{#1.pdf_tex}
}

\makeatletter
\let\ams@underbrace=\underbrace
\def\underbrace{\kernel@ifnextchar[{\underbrace@}{\underbrace@[l]}}% default value: l
\def\underbrace@[#1]#2_#3{%
  \ifx#1c\relax
    \let\ubr@align\centering%
  \else
    \ifx#1l\relax
      \let\ubr@align\raggedright%
    \else
      \ifx#1r\relax
        \let\ubr@align\raggedleft%
      \else
        \ifx#1f\relax
          \let\ubr@align\relax%
        \else
          \message{`#1' isn't a valid alignment specification for the underbrace command}%
        \fi
      \fi
    \fi
  \fi
  \setbox0=\hbox{$\displaystyle#2$}%
  \ams@underbrace{#2}_{\parbox[t]{\the\wd0}{\ubr@align#3}}%
}
\let\ubr@align\relax
\makeatother

\title{Assignment 6}
\subtitle{MATH 305 - Applied Complex Analysis}
\author{Zhi Chuen Tan (65408361)}
\date{2020W2}
\publishers{
    \includegraphics[width=0.65\textwidth]{mathlogo.eps}}

\setkomafont{pageheadfoot}{%
\normalcolor}

\newcommand{\Arg}{\text{Arg}}
\newcommand{\Log}{\text{Log}}


\ohead{MATH 305 201 \\ 2020W2}
\chead{\Large{Assignment 6}}
\ihead{Zhi Chuen Tan \\ 65408361}

\usepackage{setspace} %For line spacing

\setlength{\parindent}{0pt} % Disable indentation

\begin{document}
    \onehalfspacing
    \hypersetup{pageanchor=false}
    \begin{titlepage}
        \maketitle
        \vfill
        
    \end{titlepage}
    \hypersetup{pageanchor=true}

    \section*{Problem 1}
    The denominator can be expressed as $z^{2}-z(1+i)=(z)(z-(1+i))$, i.e. there are two singularities at 
    $z=0$, and $z=1+i$.

    \begin{enumerate}[i)]
        \item $\alpha(t) = B_{2}(1)$. We can partition $\alpha(t)$ into $\alpha_{1}(t)$, which contains
        the singularity $z=0$, and $\alpha_{2}(t)$, which contains the singularity $z=1+i$.
        \begin{align*}
            \oint_{\alpha} \frac{z-i+1}{z^{2}-z(1+i)} \, dz &= \oint_{\alpha_{1}} \frac{z-i+1}{z^{2}-z(1+i)} \, dz +
                    \oint_{\alpha_{2}} \frac{z-i+1}{z^{2}-z(1+i)} \, dz \\
                &= \oint_{\alpha_{1}} \frac{\frac{z-i+1}{z}}{z-(1+i)} \, dz +
                    \oint_{\alpha_{2}} \frac{\frac{z-i+1}{z-(1+i)}}{z} \, dz \\
        \end{align*}
        For $\oint_{\alpha_{1}} \frac{(z-i+1)/z}{z-(1+i)} \, dz$, let $f_{1}(z)=\frac{z-i+1}{z}$, 
        and $w_{1}=1+i$. Cauchy's Integral Theorem gives:
        \begin{align}
            \oint_{\alpha_{1}} \frac{\frac{z-i+1}{z}}{z-(1+i)} \, dz &= 2\pi i \, f_{1}(1+i) \nonumber\\
                &= 2\pi i (\frac{(1+i)-i+1}{1+i}) \nonumber \\
                &= 2\pi i (\frac{2}{1+i}) \nonumber\\
                &= \frac{4\pi i}{1+i} \label{eq:1.a1}
        \end{align}
        For $\oint_{\alpha_{2}} \frac{(z-i+1)/(z-(1+i))}{z} \, dz$, let $f_{2}(z)=\frac{z-i+1}{z-(1+i)}$, and
        $w_{2}=0$. Similarly, we have:
        \begin{align*}
            \oint_{\alpha_{2}} \frac{\frac{z-i+1}{z-(1+i)}}{z} \, dz &= 2\pi i \, f_{2}(0) \\
                &= 2\pi i (\frac{1-i}{-(1+i)}) \\
                &= \frac{2\pi i(i-1)}{1+i} \\
                &= \frac{2\pi (-1-i)}{1+i} \\
                &= \frac{-2\pi\cancel{(1+i)}}{\cancel{(1+i)}} \\
                &= -2\pi
        \end{align*}
        \begin{equation*}
            \therefore \oint_{\alpha} \frac{z-i+1}{z^{2}-z(1+i)} \, dz = -2\pi + \frac{4\pi i}{1+i}
        \end{equation*}
        
        \item Let $\beta=B_{1}(1+i)$. $\beta$ only contains the singularity $z=1+i$, so we can use result
        \eqref{eq:1.a1} above:
        \begin{equation*}
            \oint_{\beta} \frac{z-i+1}{z^{2}-z(1+i)} \, dz = \oint_{\alpha_{1}} \frac{\frac{z-i+1}{z}}{z-(1+i)} \, dz = \frac{4\pi i}{1+i} 
        \end{equation*}
        
        \item Let $\gamma=B_{\frac{1}{2}}(2)$. $\gamma$ does not contain either of the singularity points, so by Cauchy's Theorem:
        \begin{equation*}
            \oint_{\gamma} \frac{z-i+1}{z^{2}-z(1+i)} \, dz = 0
        \end{equation*}
    \end{enumerate}

    \section*{Problem 2}
    Let the countour be $\alpha_{R}$:
    \[
        \begin{cases}
            c_{\epsilon}(t) = \epsilon e^{i(\pi-t)}, & t\in[0,\pi], \\
            c_{1}(t) = t, & t\in[\epsilon,R], \\
            c_{R}(t) = Re^{it}, & t\in[0, \pi], \\
            c_{2}(t) = t, & t\in[-R, -\epsilon], \\
        \end{cases}    \Rightarrow
        \begin{cases}
            c'_{\epsilon}(t) = -i\epsilon e^{i(\pi-t)}, & t\in[0,\pi], \\
            c'_{1}(t) = 1, & t\in[\epsilon,R], \\
            c'_{R}(t) = iRe^{it}, & t\in[0, \pi], \\
            c'_{2}(t) = 1, & t\in[-R, -\epsilon], \\
        \end{cases}    
    \]
    As the contour is closed, by Cauchy's Theorem:
    \begin{equation}
        \oint_{\alpha_{R}} f(z) \, dz = \oint_{\alpha_{R}} \frac{e^{iz}}{z} \, dz = 0, \label{eq:2}
    \end{equation}
    where $f(z)$ can be expressed as:
    \begin{align*}
        f(z) &= \frac{e^{-y}e^{ix}}{z} \\
            &= \frac{e^{-y}(\cos(x)+i\sin(x))}{z} \\
            &= \frac{e^{-y}}{z}(\cos(x) + i\sin(x)) \\
    \end{align*}
    where $x=\Re(z), \, y=\Im(z)$. Similarly with the segments $c_{\epsilon}$ and $c_{R}$:
    \begin{align*}
        c_{\epsilon}(t) &= \epsilon e^{i\pi}e^{-it} \\
            &= -\epsilon(\cos(t) - i\sin(t)) \\
            &= -\epsilon\cos(t) + i\epsilon\sin(t), 
    \end{align*}
    \begin{align*}
        c_{R}(t) &= R(\cos(t)+i\sin(t)) \\
            &= R\cos(t) + iR\sin(t), 
    \end{align*}

    For the segment $c_{\epsilon}$:
    \begin{align}
        \int_{c_{\epsilon}} f(z) \, dz &= \int_{0}^{\pi}(\frac{e^{-\epsilon\sin(t)}}
            {\cancel{\epsilon e^{i(\pi-t)}}})(\cos(-\epsilon\cos(t)) + i\sin(-\epsilon\cos(t)))
            (-i\cancel{\epsilon e^{i(\pi-t)}}) \, dt \nonumber\\
            &= -i\int_{0}^{\pi}e^{-\epsilon\sin(t)}e^{-i\epsilon\cos(t)} \, dt \nonumber \\
            &= -i\int_{0}^{\pi}e^{-\epsilon(\sin(t)-i\cos(t))} \, dt \label{eq:2.ce} 
    \end{align}
    For the segment $c_{1}$:
    \begin{align}
        \int_{c_{1}} f(z) \, dz &= \int_{\epsilon}^{R} \frac{e^{ix}}{x} \, dx \nonumber\\
            &= \int_{\epsilon}^{R} \frac{\cos(x)}{x} + i\frac{\sin(x)}{x} \, dx \nonumber\\
            &= \int_{\epsilon}^{R} \frac{\cos(x)}{x} \, dx + i\int_{\epsilon}^{R} \frac{\sin(x)}{x} \, dx \label{eq:2.c1}
    \end{align} 
    For the segment $c_{R}$:
    \begin{align}
        \int_{c_{R}} f(z) \, dz &= \int_{0}^{\pi} (\frac{e^{R\sin(t)}}{\cancel{Re^{it}}})(\cos(R\cos(t)) + i\sin(R\cos(t)))(i\cancel{Re^{it}}) \, dt \nonumber\\
            &= i\int_{0}^{\pi}e^{R\sin(t)}e^{iR\cos(t)} \, dt \nonumber\\
            &= i\int_{0}^{\pi}e^{R(\sin(t)+i\cos(t))} \, dt \label{eq:2.cr} 
    \end{align}
    For the segment $c_{2}$:
    \begin{align}
        \int_{c_{2}} f(z) \, dz &= \int_{-R}^{-\epsilon} \frac{e^{ix}}{x} \, dx \nonumber\\
            &= \int_{-R}^{-\epsilon} \frac{\cos(x)}{x} + i\frac{\sin(x)}{x} \, dx \nonumber\\
            &= \int_{-R}^{-\epsilon} \frac{\cos(x)}{x} \, dx + i\int_{-R}^{-\epsilon} \frac{\sin(x)}{x} \, dx \label{eq:2.c2}
    \end{align} 
    For \eqref{eq:2.ce}, taking the limit as $\epsilon\to0$ gives:
    \begin{align}
        \lim_{\epsilon\to0} -i\int_{0}^{\pi}e^{-\epsilon(\sin(t)-i\cos(t))} \, dt &= -i\int_{0}^{\pi}e^{-0} \, dt \nonumber \\
            &= -i\int_{0}^{\pi} 1 \, dt \nonumber\\
            &= -\pi i \label{eq:2.limce}
    \end{align}
    Taking the limit of \eqref{eq:2.cr} as $R\to\infty$ gives:
    \begin{equation}
        \lim_{R\to\infty} i\int_{0}^{\pi}e^{R(\sin(t)+i\cos(t))} \, dt = 0 \label{eq:2.limcr}
    \end{equation}
    Taking the limit of \eqref{eq:2.c1}+\eqref{eq:2.c2} as $R\to\infty$ and $\epsilon\to0$ gives:
    \begin{align}
        &\lim_{R\to\infty}\lim_{\epsilon\to0} (\int_{\epsilon}^{R} \frac{\cos(x)}{x} \, dx + i\int_{\epsilon}^{R} \frac{\sin(x)}{x} \, dx + \int_{-R}^{-\epsilon} \frac{\cos(x)}{x} \, dx + i\int_{-R}^{-\epsilon} \frac{\sin(x)}{x} \, dx) \nonumber\\
        &= \lim_{R\to\infty}\lim_{\epsilon\to0}(\int_{\epsilon}^{R} \frac{\cos(x)}{x} \, dx + \int_{-R}^{-\epsilon} \frac{\cos(x)}{x} \, dx) + i (\lim_{R\to\infty}\lim_{\epsilon\to0}(\int_{-R}^{-\epsilon} \frac{\sin(x)}{x} \, dx)) \nonumber\\
        &= \lim_{R\to\infty}\lim_{\epsilon\to0}(\int_{\epsilon}^{R} \frac{\cos(x)}{x} \, dx + \int_{-R}^{-\epsilon} \frac{\cos(x)}{x} \, dx) + 2i\int_{0}^{\infty}\frac{\sin(x)}{x}\, dx, \, \text{by Hint (a)} \label{2.limc1c2}
    \end{align}
    Combining \eqref{eq:2.limce}, \eqref{eq:2.limcr}, \eqref{2.limc1c2}, and substituting into \eqref{eq:2} then gives:
    \begin{align*}
        -\pi i + 0 + \lim_{R\to\infty}\lim_{\epsilon\to0}(\int_{\epsilon}^{R} \frac{\cos(x)}{x} \, dx + \int_{-R}^{-\epsilon} \frac{\cos(x)}{x} \, dx) + 2i\int_{0}^{\infty}\frac{\sin(x)}{x}\, dx &= 0 \\
        \lim_{R\to\infty}\lim_{\epsilon\to0}(\int_{\epsilon}^{R} \frac{\cos(x)}{x} \, dx + \int_{-R}^{-\epsilon} \frac{\cos(x)}{x} \, dx) + 2i\int_{0}^{\infty}\frac{\sin(x)}{x}\, dx &= \pi i
    \end{align*}
    Taking Im(LHS) = Im(RHS) gives:
    \begin{align*}
        2\int_{0}^{\infty}\frac{\sin(x)}{x}\, dx &= \pi \\
        \int_{0}^{\infty}\frac{\sin(x)}{x}\, dx &= \frac{\pi}{2} 
    \end{align*} \qed
    
    \section*{Problem 3}
    Let the countour $\gamma(t)$ be parametrized as follows:
    \[
        \begin{cases}
            \gamma_{1}(t) = t, & t\in[0,R], \\
            \gamma_{2}(t) = R(1+it), & t\in[0,a], \\
            \gamma_{3}(t) = t(1 + ia), & t\in[0,R], \\
        \end{cases}    \Rightarrow
        \begin{cases}
            \gamma_{1}'(t) = 1, \\
            \gamma_{2}'(t) = Ri, \\
            \gamma_{3}'(t) = (1+ia), \\
        \end{cases}
    \]
    Let $f(z)=e^{-z^{2}}$. As $\gamma(t)$ is closed, by Cauchy's Theorem we have $\oint_{\gamma} f(z) \, dz = 0 $:
    \begin{equation}
        \Rightarrow \int_{0}^{R} e^{-t^{2}} \, dt + \int_{0}^{a} e^{-R^{2}(1+it)^{2}}(Ri) \, dt \, \underbrace[c]{-\int_{0}^{R} e^{-(t(1+ia)^{2})}(1+ia) \, dt}_{We subtract this term as the orientation is opposite of the parametrization above} = 0 \label{eq:3.1}
    \end{equation}
    Taking the limit of the left hand side of \eqref{eq:3.1} as $R\to\infty$ gives:
    \begin{align*}
        \int_{0}^{\infty} e^{-t^{2}} \, dt + \int_{0}^{a} e^{-\infty}(Ri) \, dt - \int_{0}^{\infty} e^{-(t(1+ia)^{2})}(1+ia) \, dt &= \frac{\sqrt{\pi}}{2} + 0 - (1+ia)\int_{0}^{\infty} e^{-(1+ia)^{2}t^{2}} \, dt
    \end{align*}
    Substituting this result back into \eqref{eq:3.1} gives:
    \begin{align*}
        \frac{\sqrt{\pi}}{2} - (1+ia)\int_{0}^{\infty} e^{-(1+ia)^{2}t^{2}} \, dt &= 0 \\
        (1+ia)\int_{0}^{\infty} e^{-(1+ia)^{2}t^{2}} \, dt &= \frac{\sqrt{\pi}}{2} \\
        \int_{0}^{\infty} e^{-(1+ia)^{2}t^{2}} \, dt &= \frac{\sqrt{\pi}}{2(1+ia)} 
    \end{align*} \qed \\
    From this:
    \begin{gather*}
        \int_{0}^{\infty} e^{-(1+2ia-a^{2})t^{2}} \, dt = \frac{\sqrt{\pi}}{2(1+ia)} \\
        \int_{0}^{\infty} e^{-((1-a^{2})t^{2}+2iat^{2})} \, dt = \frac{\sqrt{\pi}}{2(1+ia)} \\
        \int_{0}^{\infty} e^{(a^{2}-1)t^{2}-2iat^{2}} \, dt = \frac{\sqrt{\pi}}{2}(\frac{1-ia}{1+a^{2}}) \\
        \int_{0}^{\infty} e^{(a^{2}-1)t^{2}}e^{-i2at^{2}} \, dt = \frac{\sqrt{\pi}}{2(1+a^{2})} - i\frac{a\sqrt{\pi}}{2(1+a^{2})} \\
        \int_{0}^{\infty} e^{(a^{2}-1)t^{2}}(\cos(2at^{2})-i\sin(2at^{2})) \, dt = \frac{\sqrt{\pi}}{2(1+a^{2})} - i\frac{a\sqrt{\pi}}{2(1+a^{2})} \\
    \end{gather*}
    Taking Re(LHS)=Re(RHS):
    \begin{equation*}
        \int_{0}^{\infty} e^{(a^{2}-1)t^{2}}\cos(2at^{2}) \, dt = \frac{\sqrt{\pi}}{2(1+a^{2})} 
    \end{equation*}
    Let $a=1$:
    \begin{gather*}
        \int_{0}^{\infty} e^{(1-1)t^{2}}\cos(2t^{2}) \, dt = \frac{\sqrt{\pi}}{2(1+1)} \\
        \int_{0}^{\infty} \cos(2t^{2}) \, dt = \frac{\sqrt{\pi}}{4} 
    \end{gather*}
    Let $u=\sqrt{2}t$, $du = \sqrt{2}\, dt$:
    \begin{gather*}
        \therefore \int_{0}^{\infty} \cos(2(\frac{u}{\sqrt{2}})^{2}) \, \frac{du}{\sqrt{2}} = \frac{\sqrt{\pi}}{4} \\
        \int_{0}^{\infty} \cos(2(\frac{u^{2}}{2})) \, \frac{du}{\sqrt{2}} = \frac{\sqrt{\pi}}{4} \\
        \frac{1}{\sqrt{2}}\int_{0}^{\infty} \cos(u^{2}) \, du= \frac{\sqrt{\pi}}{4} \\
        \int_{0}^{\infty} \cos(u^{2}) \, du= \frac{\sqrt{2\pi}}{4} \\
        \int_{0}^{\infty} \cos(u^{2}) \, du= \sqrt{\frac{\cancel{2}\pi}{\cancelto{8}{16}}} \\
        \int_{0}^{\infty} \cos(u^{2}) \, du= \sqrt{\frac{\pi}{8}} \\
        \Rightarrow \int_{0}^{\infty} \cos(t^{2}) \, dt= \sqrt{\frac{\pi}{8}} 
    \end{gather*} \qed

    Taking Im(LHS)=Im(RHS):
    \begin{equation*}
        \int_{0}^{\infty} e^{(a^{2}-1)t^{2}}\sin(2at^{2}) \, dt = \frac{a\sqrt{\pi}}{2(1+a^{2})}
    \end{equation*}
    Let $a=1$:
    \begin{gather*}
        \int_{0}^{\infty} e^{(1-1)t^{2}}\sin(2t^{2}) \, dt = \frac{\sqrt{\pi}}{2(1+1)} \\
        \int_{0}^{\infty} \sin(2t^{2}) \, dt = \frac{\sqrt{\pi}}{4} 
    \end{gather*}    
    Similarly, let $u=\sqrt{2}t$, $du = \sqrt{2}\, dt$:
    \begin{gather*}
        \frac{1}{\sqrt{2}}\int_{0}^{\infty} \sin(u^{2}) \, du = \frac{\sqrt{\pi}}{4} \\
        \int_{0}^{\infty} \sin(u^{2}) \, du = \frac{\sqrt{2\pi}}{4} \\
        \int_{0}^{\infty} \sin(u^{2}) \, du = \sqrt{\frac{\cancel{2}\pi}{\cancelto{8}{16}}} \\
        \int_{0}^{\infty} \sin(u^{2}) \, du = \sqrt{\frac{\pi}{8}} \\
        \Rightarrow \int_{0}^{\infty} \sin(t^{2}) \, dt = \sqrt{\frac{\pi}{8}} = \int_{0}^{\infty} \cos(t^{2}) \, dt 
    \end{gather*} \qed \\
    In both cases, we treat $u$ as a 'dummy variable' and just substitute $u=t$ back into the expression.

    \section*{Problem 4}
    Since $f(z)=g(z)\, \forall z \in\alpha$, we have that $|f(z)-g(z)|=0$ on the curve $\alpha$. By the Maximum Modulus Principle, 
    $|f(z)-g(z)|$ reaches its maximum on the boundary of the bounded domain (in this case $\alpha$). Therefore, there is no point in the
    interior of $\alpha$ such that $|f(z)-g(z)|>0$. As the modulus of any complex number cannot be negative, this means that 
    \begin{equation*}
        |f(z)-g(z)|=0 \Rightarrow f(z)=g(z), \, \forall z\in \text{int}(\alpha)
    \end{equation*}\qed
    
\end{document}  
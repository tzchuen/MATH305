\documentclass[a4paper, titlepage, DIV=14]{scrartcl}
\usepackage{import}
\usepackage{xifthen}
\usepackage{pdfpages}
\usepackage{transparent}
\usepackage{amsmath,amssymb,amsthm}
\usepackage{bm}
\usepackage{biblatex}
\usepackage{enumerate}
\usepackage{caption}
\usepackage{float}
\usepackage[colorlinks=true, allcolors=black, urlcolor=cyan]{hyperref}
\usepackage{graphicx}
\usepackage[framed,numbered]{matlab-prettifier}
\usepackage{mathtools}
%\usepackage{mcode}
\usepackage[headsepline]{scrlayer-scrpage}
\usepackage{graphicx}
\usepackage{physics}
\usepackage{siunitx}
\usepackage{cancel}
\usepackage{appendix}
\usepackage{lipsum}
\usepackage{listings}
\usepackage{fontawesome}

\newcommand{\incfig}[1]{%
    \def\svgwidth{0.75\columnwidth}
    \import{./}{#1.pdf_tex}
}

\title{Assignment 4}
\subtitle{MATH 305 - Applied Complex Analysis}
\author{Zhi Chuen Tan (65408361)}
\date{2020W2}
\publishers{
    \includegraphics[width=0.65\textwidth]{mathlogo.eps}}

\setkomafont{pageheadfoot}{%
\normalcolor}

\newcommand{\Arg}{\text{Arg}}
\newcommand{\Log}{\text{Log}}


\ohead{MATH 305 201 \\ 2020W2}
\chead{\Large{Assignment 4}}
\ihead{Zhi Chuen Tan \\ 65408361}

\usepackage{setspace} %For line spacing

\setlength{\parindent}{0pt} % Disable indentation

\begin{document}
    \onehalfspacing
    \hypersetup{pageanchor=false}
    \begin{titlepage}
        \maketitle
        \vfill
        
    \end{titlepage}
    \hypersetup{pageanchor=true}

    % \section*{Problem 1}
    % \begin{enumerate}[(i)]
    %     \item 
    %     \item 
    %     \item 
    % \end{enumerate}

    \section*{Problem 2}
    We assume the ice block has infinite length and lies on the complex plane as shown in \autoref{fig:2}.
    \begin{figure}[!h]
        \centering
        \incfig{2}
        \caption{}
        \label{fig:2}
    \end{figure} \\
    Let $\phi(x,y) = A\ln|z| + iB\Arg(z)$. Applying the 
    boundary conditions at $\phi=T$ and $\phi=0$, gives:
    \begin{align*}
        &\begin{cases}
            A\ln(\frac{1}{2}\sigma) + iB\frac{\pi}{2} = T \\
            A\ln(\frac{1}{2}\sigma) - iB\frac{\pi}{2} = 0
        \end{cases} \\
        &\Rightarrow
        \begin{cases}
            A = \frac{T}{2\ln(\frac{1}{2}\sigma)} \\
            B = -\frac{T}{\pi}i
        \end{cases} \\
    \end{align*}
    This gives the temperature distribution:
    \begin{equation*}
        \phi(z) = \frac{T}{2\ln(\frac{1}{2}\sigma)}\ln|z| + \frac{T}{\pi}\Arg(z)
    \end{equation*}

    % We assume the ice block is of infinite length, and the bottom layer (i.e. the face of the ice 
    % that faces the water) lies on the line $\{z\in \mathbb{C} \, : \Im(z) = 0\}$, and the top layer 
    % (i.e. the face of the ice that faces the air) lines on the line 
    % $\{z\in \mathbb{C} \, : \, \Im(z) = \sigma\}$. We also assume that the temperature distribution
    % is uniform throughout the ice layer.\\
    
    % \par We also know that the temperature distribution
    % of this ice of uniform thickness, $T(x,y) = u(x,y) + iv(x,y)$, will be a harmonic function, i.e.
    % $\Delta u = \Delta v = 0$. For $u(x,y)$:
    % \begin{equation*}
    %     \Delta u = \partial^{2}_{xx} + \partial^{2}_{yy} = 0 
    % \end{equation*}
    % Taking the antiderivative of $\partial^{2}_{xx}$ with respect to $x$ twice:
    % \begin{align*}
    %     \partial_{x} &= 0 + A(y) \\
    %     u &= A(y)x + B(y) \\
    % \end{align*}
    % Similarly:
    % \begin{align*}
    %     \partial_{y} &= 0 + C(x) \\
    %     u &= C(x)y + D(x) \\
    %     &= A(y)x + B(y) \\
    %     &\Rightarrow C(x) = x, \, A(y) = y, \, B(y) = D(x) = c, \, c \in \mathbb{R} \\
    %     \Rightarrow u &= xy + c
    % \end{align*}
    % As $T(x,y)$ is harmonic, then $T\in H(\Omega)$, which means that the Cauchy-Riemann equations 
    % are satisfied, giving $\partial_{y} \, v(x,y) = \partial_{x} \, u(x,y) = y$. Taking the antiderivative
    % with respect to y:
    % \begin{equation}
    %     v(x,y) = \frac{1}{2}y^{2} + P(x) \label{eq:1}
    % \end{equation}
    % The 2nd Cauchy-Riemann equation asserts that $\partial_{x} \, v(x,y) = -\partial_{y} \, u(x,y) = 
    % -x$. Taking the partial of \autoref{eq:1} with respect to $x$:
    % \begin{align*}
    %     P'(x) &= -x \\
    %     \Rightarrow P(x) &= -\frac{1}{2}x^{2} + d, \, d\in \mathbb{R}
    % \end{align*}
    % This gives:
    % \begin{equation*}
    %     v(x,y) = \frac{1}{2}y^{2} - \frac{1}{2}x^{2} + d,
    % \end{equation*} and also that
    % \begin{equation*}
    %     T(x,y) = (xy + c) + i(\frac{1}{2}y^{2} - \frac{1}{2}x^{2} + d)
    % \end{equation*}

    % Based on the assumption of a uniform temperature distribution, we can plug in the boundary conditions
    % at $x=0$. This gives us:
    % \begin{align*}
    %     T(0,0) &= c+di = 0 \\
    %     T(0,\sigma) &= \frac{1}{2}\sigma^{2} + c + di = T \\
    %     \Rightarrow&
    %     \begin{cases}
    %         c + di = 0 \\
    %         \frac{1}{2}\sigma^{2} + c + di = T \\
    %     \end{cases} \\
    %     \Rightarrow&
    %     \begin{cases}
    %        123
    %     \end{cases} \\
    % \end{align*}



    \section*{Problem 3}
    \begin{enumerate}[(i)]
        \item $\Log(\frac{z-1}{z-2})$ has a branch cut on:
        \[
            \begin{cases}
                \Im(\frac{z-1}{z-2}) = 0 \\
                \Re(\frac{z-1}{z-2}) \leq 0
            \end{cases}    
        \]
        \begin{align*}
            \Rightarrow \frac{z-1}{z-2} &= ((x-1)+yi)((x-2)+yi)^{-1} \\
                &= \frac{((x-1)+yi)((x-2)-yi)}{(x-2)^{2}+y^{2}} \\
                &= \frac{(x-1)(x-2) - iy(x-1) + iy(x-2) + y^{2}}{(x-2)^{2}+y^{2}} \\
                &= \frac{x^{2}-3x+2+y^{2}}{(x-2)^{2}+y^{2}} + iy\frac{(x-1)-(x-2)}{(x-2)^{2}+y^{2}} \\
                &= \frac{x^{2}-3x+2+y^{2}}{(x-2)^{2}+y^{2}} + i\frac{y}{(x-2)^{2}+y^{2}} \\
        \end{align*}
        This gives the system of equations:
          \begin{align*}
              &\begin{cases}
                y = 0 \\
                x^{2}-3x+2+y^{2} \leq 0
              \end{cases} \\
              &\Rightarrow x^{2}-3x+2 \leq 0 \\
              &(x-1)(x-2) \leq 0 \\
              & 1 \leq x \leq 2
          \end{align*} 
        $\therefore \Omega = \mathbb{C} \, \backslash \, \{\Re(z) \in [1,2]\}$
        
        \item We can write:
        \begin{align*}
            (z^{2}-1)^{\frac{1}{2}} &= (-1)^{\frac{1}{2}}(1-z^{2})^{\frac{1}{2}} \\
                &= i(1-z^{2})^{\frac{1}{2}} \\
                &= ie^{\frac{1}{2}\Log(1-z^{2})},
        \end{align*}which has a branch cut on the set:
        \[
          \begin{cases}
              \Im(1-z^{2}) = 0 \\
              \Re(1-z^{2}) \leq 0
          \end{cases}  
        \]
        We have that:
        \begin{align*}
            1-z^{2} &= 1 - (x+yi)^{2} \\
                &= 1 - (x^{2}+2xyi-y^{2}) \\
                &= 1 - x^{2} + y^{2} - 2xyi \\
                &\Rightarrow \begin{cases}
                    -2xy = 0 \\
                    1 - x^{2} + y^{2} \leq 0 \\
                \end{cases},
        \end{align*}which indicates $y=0$ or $x=0$. If $x=0$:
        \begin{align*}
            1 + y^{2} &\leq 0 \\
            y^{2} &\leq -1,
        \end{align*} which cannot be true, as $y\in \mathbb{R}$. Therefore, $y$ must be 0:
        \begin{align*}
            1-x^{2} &\leq 0 \\
            x^{2} &\geq 1 \\
            &\Rightarrow x \leq -1, \, x \geq 1,
        \end{align*}which lies on the beyond the unit circle centered about 0 in $\mathbb{C}$. Therefore,
        any branch of $(z^{2}-1)^{\frac{1}{2}}$ within the disc $B_{1}(0)$ is holomorphic in $B_{1}(0)$,
        e.g. the branch $\{\Im(z)\in (-1, 1)\}$.
    \end{enumerate}

    \section*{Problem 4}
    \begin{enumerate}[(i)]
        \item The contour can be parametrized as $\alpha = \alpha_{1} + \alpha_{2} + \alpha_{3} + \alpha_{4}
        + \alpha_{5}$, where the initial point of $\alpha_{j}$ is the end-point of $\alpha_{j-1}$:
        \[
          \begin{cases}
              \alpha_{1} = re^{i(\pi-t)}, & t = [0, \pi], \\
              \alpha_{2} = t, & t = [r, R], \\
              \alpha_{3} = Re^{it}, & t = [0, \pi], \\
              \alpha_{4} = t, & t = [-R,-r], \\
          \end{cases}  
        \]

        \item In polar coordinates, we have that $x=r\cos(\theta)$ and $y=r\sin(\theta)$:
        \begin{gather*}
            \frac{x^{2}}{a^{2}} + \frac{y^{2}}{b^{2}} = 1 
                \Leftrightarrow \frac{r^{2}\cos^{2}(\theta)}{a^{2}} + 
                    \frac{r^{2}\sin^{2}(\theta)}{b^{2}} = 1 \\
                \frac{a^{2}r^{2}\sin^{2}(\theta) + b^{2}r^{2}\cos^{2}(\theta)}{a^{2}b^{2}} = 1 \\
                a^{2}r^{2}\sin^{2}(\theta) + b^{2}r^{2}\cos^{2}(\theta) = a^{2}b^{2} \\
                r^{2}(a^{2}\sin^{2}(\theta) + b^{2}\cos^{2}(\theta)) = a^{2}b^{2} \\
                r^{2} = \frac{a^{2}b^{2}}{a^{2}\sin^{2}(\theta) + b^{2}\cos^{2}(\theta)} \\
                r = \sqrt{\frac{a^{2}b^{2}}{a^{2}\sin^{2}(\theta) + b^{2}\cos^{2}(\theta)}} \\
        \end{gather*}
        The contour can therefore be parametrized as:
        \begin{equation*}
            \alpha(t) = \frac{ab}{\sqrt{a^{2}\sin^{2}(t) + b^{2}\cos^{2}(t)}}, \, t\in[0,2\pi]
        \end{equation*}
    \end{enumerate}
\end{document}
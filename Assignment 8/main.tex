\documentclass[letterpaper, titlepage, DIV=14]{scrartcl}
\usepackage{import}
\usepackage{xifthen}
\usepackage{pdfpages}
\usepackage{transparent}
\usepackage{amsmath,amssymb,amsthm}
\usepackage{bm}
%\usepackage{biblatex}
\usepackage{enumerate}
\usepackage{caption}
\usepackage{float}
\usepackage[colorlinks=true, allcolors=black, urlcolor=cyan]{hyperref}
\usepackage{graphicx}
\usepackage[framed,numbered]{matlab-prettifier}
\usepackage{mathtools}
%\usepackage{mcode}
\usepackage[headsepline]{scrlayer-scrpage}
\usepackage{graphicx}
\usepackage{physics}
\usepackage{siunitx}
\usepackage{cancel}
\usepackage{appendix}
\usepackage{lipsum}
\usepackage{listings}
%\usepackage{fontawesome}

\newcommand{\incfig}[1]{%
    \def\svgwidth{0.75\columnwidth}
    \import{./}{#1.pdf_tex}
}

\makeatletter
\let\ams@underbrace=\underbrace
\def\underbrace{\kernel@ifnextchar[{\underbrace@}{\underbrace@[l]}}% default value: l
\def\underbrace@[#1]#2_#3{%
  \ifx#1c\relax
    \let\ubr@align\centering%
  \else
    \ifx#1l\relax
      \let\ubr@align\raggedright%
    \else
      \ifx#1r\relax
        \let\ubr@align\raggedleft%
      \else
        \ifx#1f\relax
          \let\ubr@align\relax%
        \else
          \message{`#1' isn't a valid alignment specification for the underbrace command}%
        \fi
      \fi
    \fi
  \fi
  \setbox0=\hbox{$\displaystyle#2$}%
  \ams@underbrace{#2}_{\parbox[t]{\the\wd0}{\ubr@align#3}}%
}
\let\ubr@align\relax
\makeatother

\title{Assignment 8}
\subtitle{MATH 305 - Applied Complex Analysis}
\author{Zhi Chuen Tan (65408361)}
\date{2020W2}
\publishers{
    \includegraphics[width=0.65\textwidth]{mathlogo.eps}}

\setkomafont{pageheadfoot}{%
\normalcolor}

\newcommand{\Arg}{\text{Arg}}
\newcommand{\Log}{\text{Log}}


\ohead{MATH 305 201 \\ 2020W2}
\chead{\Large{Assignment 8}}
\ihead{Zhi Chuen Tan \\ 65408361}

\usepackage{setspace} %For line spacing

\setlength{\parindent}{0pt} % Disable indentation

\begin{document}
    \onehalfspacing
    \hypersetup{pageanchor=false}
    \begin{titlepage}
        \maketitle
        \vfill
        
    \end{titlepage}
    \hypersetup{pageanchor=true}

    \section*{Problem 1}
    The Taylor series is given by:
    \begin{equation*}
      f(w) = \sum_{n=0}^{\infty}\frac{f^{(n)}(z_{0})}{n!}(w-z_{0})^{n},
    \end{equation*} where $w \in \mathbb{C}, \, z_{0} \in \mathbb{C}$
    \begin{enumerate}[i)]
      \item 
      \begin{align*}
        \frac{1+z}{1-z} &= \frac{1+i+(z-i)}{1-i-(z-i)} \\
          &= \frac{1+i+(z-i)}{(1-i)(1 - \frac{z-i}{1-i})} \\
          &= \frac{\frac{1+i+(z-i)}{(1-i)}}{1 - \frac{z-i}{1-i}}
        % For $\frac{1}{1-z}$, the Taylor series is:
        % \begin{align*}
        %   f_{1}(z) = \frac{1}{1-z} &= \frac{1}{1-z_{0}} + \frac{1}{(1-z_{0})^{2}}(z-z_{0}) + \frac{1}{(1-z_{0})^{3}}(z-z_{0})^{2} + \ldots \\
        %     &= \frac{1}{1-i} + \frac{1}{(1-i)^{2}}((i + (z-i)) - i) + \frac{1}{(1-i)^{3}}((i+(z-i))-i)^{2} + \ldots \\
        %     &= \frac{1}{1-i} + \frac{1}{(1-i)^{2}}(z-i) + \frac{1}{(1-i)^{3}}(z-i)^{2} + \ldots \\
        %     &= \sum_{n=0}^{\infty}\frac{(z-i)^{n}}{(1-i)^{n+1}} \\
        %     &= \frac{1}{1-i}\sum_{n=0}^{\infty}(\frac{z-i}{1-i})^{n}
        % \end{align*}
        % Therefore the Taylor series for the given function is:
        % \begin{equation*}
        %   f(z) = \frac{1+z}{1-z} = \frac{1+z}{1-i}\sum_{n=0}^{\infty}(\frac{z-i}{1-i})^{n}
        % \end{equation*}
        % $\frac{1}{f(z)} = \frac{1-z}{1+z}$ has a simple zero at $z=1$, which means $f(z)$ has a simple pole at $z=1$. The radius of convergence is
        % therefore:
        % \begin{align*}
        %   |i-1| &= \sqrt{1^{2}+1^{2}} \\
        %     &= \sqrt{2}
        % \end{align*}
      \end{align*}
      Using the geometric series then gives:
      \begin{align*}
        \frac{1+z}{1-z} &= \frac{1+i+(z-i)}{(1-i)}\sum_{n=0}^{\infty}(\frac{z-i}{1-i})^{n} \\
          &= \frac{1+i}{1-i}\frac{z-i}{1-i}\sum_{n=0}^{\infty}(\frac{z-i}{1-i})^{n} \\
          &= \frac{1+i}{1-i}\sum_{n=0}^{\infty}(\frac{z-i}{1-i})^{n+1} \\
      \end{align*}
      The function $(\frac{1+z}{1-z})^{-1}=\frac{1-z}{1+z}$ has a simple zero at $z=1$. Therefore, $\frac{1+z}{1-z}$ has a simple 
      pole at $z=1$, which gives a radius of convergence:
      \begin{align*}
        |i-1| &= \sqrt{1^{2}+1^{2}} \\
          &= \sqrt{2}
      \end{align*}
      
      \item We have (from lectures) that the Taylor series for trigonometric functions for the complex trigonometric functions are 
      equivalent to those from the reals. Therefore:
      \begin{align*}
        z^{4}\cos(3z) &= z^{4}\sum_{n=0}^{\infty}\frac{(-1)^{n}}{(2n)!}(3z)^{2n}
      \end{align*}
      As the function $1/z^{4}\cos(3z)$ has a zero only at infinity, the function is entire and the radius of convergence is infinite.
     
    \end{enumerate}
    
    \section*{Problem 2}
    Let:
    \begin{equation*}
      f(z) = \sum_{n=0}^{\infty}\frac{f^{(n)}(z_{0})}{n!}(z-z_{0})^{n} = \sum_{n=0}^{\infty}a_{n}(z-z_{0})^{n}, \, a_{n} = \frac{f^{(n)}(z_{0})}{n!}
    \end{equation*}
    Let also $z_{0}=0$. We then have:
    \begin{align*}
      f(z)  &= \sum_{n=0}^{\infty}a_{n}(z-z_{0})^{n} \\
            &= \sum_{n=0}^{\infty}a_{n}z^{n} \\
            &= a_{0} + a_{1}z + \sum_{n=2}^{\infty}a_{n}z^{n} \\
      zf'(z) &= z\sum_{n=1}^{\infty}na_{n}(z-z_{0})^{n-1} \\
            &= z\sum_{n=1}^{\infty}na_{n}z^{n-1} \\
            &= \sum_{n=1}^{\infty}na_{n}z^{n} \\
            &= a_{1} + \sum_{n=2}^{\infty}na_{n}z^{n} \\
      f''(z) &= \sum_{n=2}^{\infty}n(n-1)a_{n}(z-z_{0})^{n-2} \\
            &= \sum_{n=2}^{\infty}n(n-1)a_{n}z^{n-2} \\
            &= \sum_{n=0}^{\infty}(n+2)(n+1)a_{n+2}z^{n} \\
            &= 2a_{0} + 6a_{1} + \sum_{n=2}^{\infty}(n+2)(n+1)a_{n+2}z^{n} \\
    \end{align*}
    Where:
    \[
      \begin{cases}
        a_{0} = \frac{f(0)}{0!} = 1 \\
        a_{1} = \frac{f'(0)}{1!} = 0
      \end{cases}
    \]
    Plugging these back into the differential equation gives:
    \begin{gather*}
      2 + \sum_{n=2}^{\infty}(n+2)(n+1)a_{n+2}z^{n} - \sum_{n=2}^{\infty}na_{n}z^{n} - 1 - \sum_{n=2}^{\infty}a_{n}z^{n} = 0 \\
      1 + \sum_{n=2}^{\infty}((n+2)(n+1)a_{n+2}-(n+1)a_{n})z^{n} = 0 
    \end{gather*}
    By comparing coefficients, we obtain the recursion relation:
    \begin{align*}
      (n+2)(n+1)a_{n+2}-(n+1)a_{n} &= 0 \\
      (n+2)\cancel{(n+1)}a_{n+2} &= \cancel{(n+1)}a_{n} \\
      a_{n+2} &= \frac{a_{n}}{n+2}
    \end{align*}

    Given that $a_{0}=1, \, a_{1} = 0$, and the above result, we have:
    \begin{align*}
      a_{2} &= \frac{a_{0}}{2} = \frac{1}{2} \\
      a_{3} &= \frac{a_{1}}{3} = 0 \\
      a_{4} &= \frac{a_{2}}{4} = \frac{1}{8}
    \end{align*}
    We can observe that, as all odd powered terms $a_{2k+1}=a_{2k-1}/(2k-1)$, $\forall k \in \mathbb{Z}, \, k \geq1$ (i.e. each odd numbered term
    is a factor of the previous odd powered term), as the first odd powered term $a_{1}=0$, all subsequent odd powered terms will be $0$. Meanwhile,
    the even powered terms have the recursion relation:
    \begin{align*}
      a_{2k+2} &= \frac{a_{2k}}{2k+2} \\
        &= \frac{a_{2k-2}}{(2k+2)(2k-2)} \\
        &= \frac{a_{2k-4}}{(2k+2)(2k-2)(2k-4)} \\
        &= \ldots 
    \end{align*}
    Let $j=k+1$. Then:
    \begin{align*}
      a_{2k+2} = a_{2j} &= \frac{a_{2j-2}}{2j-2} \\
        &= \frac{a_{2j-4}}{(2j-2)(2j-4)} \\
        &= \frac{a_{2j-6}}{(2j-2)(2j-4)(2j-6)} \\
        &= \ldots \\
        &= \frac{a_{((2j)!!)}}{(2j)!!}
    \end{align*}
    As the first even powered term has a numerator of 1, all subsequent even powered terms will have a numerator of 1 as well. 
    Therefore we can rewrite the Maclaurin series of $f(z)$ as:
    \begin{align*}
      f(z) &= 1 + 0 + \underbrace[c]{\sum_{n=2}^{\infty}\frac{1}{(2n)!!}z^{2n}}_{we can use $2n$ as only the even power terms $\neq0$} \\
        &= 1 + \underbrace[c]{\sum_{n=1}^{\infty}\frac{1}{(2n)!!}z^{2n}}_{we can start the series from $n=1$ as $a_{n}=0$} \\
    \end{align*} \qed

    

    \section*{Problem 3}
    We can rewrite $f(z)$ as:
    \begin{align*}
      f(z) &= \frac{1}{1-(-\Log(1-z))} \\
          &= \sum_{n=0}^{\infty}(-1)^{n}(\Log(1-z))^{n},
    \end{align*} using the geometric series.
    
\end{document}  
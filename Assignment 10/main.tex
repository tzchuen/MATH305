\documentclass[letterpaper, titlepage, DIV=14]{scrartcl}
\usepackage{import}
\usepackage{xifthen}
\usepackage{pdfpages}
\usepackage{transparent}
\usepackage{amsmath,amssymb,amsthm}
\usepackage{bm}
%\usepackage{biblatex}
\usepackage{enumerate}
\usepackage{caption}
\usepackage{float}
\usepackage[colorlinks=true, allcolors=black, urlcolor=cyan]{hyperref}
\usepackage{graphicx}
\usepackage[framed,numbered]{matlab-prettifier}
\usepackage{mathtools}
%\usepackage{mcode}
\usepackage[headsepline]{scrlayer-scrpage}
\usepackage{graphicx}
\usepackage{physics}
\usepackage{siunitx}
\usepackage{cancel}
\usepackage{appendix}
\usepackage{lipsum}
\usepackage{listings}
%\usepackage{fontawesome}

\newcommand{\incfig}[1]{%
    \def\svgwidth{0.75\columnwidth}
    \import{./}{#1.pdf_tex}
}

\makeatletter
\let\ams@underbrace=\underbrace
\def\underbrace{\kernel@ifnextchar[{\underbrace@}{\underbrace@[l]}}% default value: l
\def\underbrace@[#1]#2_#3{%
  \ifx#1c\relax
    \let\ubr@align\centering%
  \else
    \ifx#1l\relax
      \let\ubr@align\raggedright%
    \else
      \ifx#1r\relax
        \let\ubr@align\raggedleft%
      \else
        \ifx#1f\relax
          \let\ubr@align\relax%
        \else
          \message{`#1' isn't a valid alignment specification for the underbrace command}%
        \fi
      \fi
    \fi
  \fi
  \setbox0=\hbox{$\displaystyle#2$}%
  \ams@underbrace{#2}_{\parbox[t]{\the\wd0}{\ubr@align#3}}%
}
\let\ubr@align\relax
\makeatother

\title{Assignment 10}
\subtitle{MATH 305 - Applied Complex Analysis}
\author{Zhi Chuen Tan (65408361)}
\date{2020W2}
\publishers{
    \includegraphics[width=0.65\textwidth]{mathlogo.eps}}

\setkomafont{pageheadfoot}{%
\normalcolor}

\newcommand{\Arg}{\text{Arg}}
\newcommand{\Log}{\text{Log}}


\ohead{MATH 305 201 \\ 2020W2}
\chead{\Large{Assignment 10}}
\ihead{Zhi Chuen Tan \\ 65408361}

\usepackage{setspace} %For line spacing

\setlength{\parindent}{0pt} % Disable indentation

\begin{document}
    \onehalfspacing
    \hypersetup{pageanchor=false}
    \begin{titlepage}
        \maketitle
        \vfill
        
    \end{titlepage}
    \hypersetup{pageanchor=true}

    \section*{Problem 1}
    Let:
    \begin{align*}
      \sin(\varphi) &= \frac{1}{2i}(z-\frac{1}{z}), \, z = e^{i\varphi}, \, dz = ie^{i\varphi} d\varphi= iz \, d\varphi \\
      \cos(\varphi) &= \frac{1}{2}(z+\frac{1}{z}), \, z = e^{i\varphi}, \, dz = ie^{i\varphi} d\varphi= iz \, d\varphi \\
    \end{align*}
    
    The real integral can then be rewritten as:
    \begin{align*}
      I &= \oint_{|z|=1} \frac{(\frac{1}{2i}(z-\frac{1}{z}))^{2}}{5+4(\frac{1}{2}(z+\frac{1}{z}))} \, \frac{dz}{iz} \\
        &= \oint_{|z|=1} \frac{\frac{1}{4(i)^{2}}(z-\frac{1}{z})^{2}}{5+2(z+\frac{1}{z})} \, \frac{dz}{iz} \\
        &= -\frac{1}{4i}\oint_{|z|=1} \frac{(z-\frac{1}{z})^{2}}{z(5+2(z+\frac{1}{z}))} \, dz \\
        &= -\frac{1}{4i}\oint_{|z|=1} \frac{(\frac{1}{z})^{2}(z^{2}-1)^{2}}{z(5+2z+\frac{2}{z})} \, dz \\
        &= -\frac{1}{4i}\oint_{|z|=1} \frac{(z^{2}-1)^{2}}{2z^{2}(z+\frac{1}{2})(z+2)} \, dz \\
    \end{align*}
    Let $f(z) = \frac{(z^{2}-1)^{2}}{2z^{2}(z+\frac{1}{2})(z+2)}$. We can observe that $f$ has simple poles $-\frac{1}{2}$ and $-2$, and a pole of order 2 at $z_{0}=0$ As the pole $z_{0}=-2$ lies outside the unit circle, we only have to consider the Residue of $f$ at the simple pole at $-\frac{1}{2}$ and double pole at $z_{0}=0$ :
    \begin{align*}
      \Res(f \, ; \, -\frac{1}{2}) &= \lim_{z\to -\frac{1}{2}}(z+\frac{1}{2})f(z) \\
       &= \lim_{z\to -\frac{1}{2}} \frac{(z^{2}-1)^{2}}{2z^{2}(z+2)} \\
       &= \frac{3}{4}
    \end{align*}
    \begin{align*}
      \Res(f \, ; \, 0) &= \lim_{z\to 0}\frac{d}{dz}z^{2}f(z) \\ 
        &= \lim_{z\to 0} \frac{d}{dz} \frac{(z^{2}-1)^{2}}{2(z+\frac{1}{2})(z+2)} \\
        &= \lim_{z\to 0} \frac{4z^{5}+15z^{4}+8z^{3}-10z^{2}-12z-5}{(z+2)^{2}(2z+1)^{2}}
    \end{align*}
    \begin{equation*}
      \Res(f \, ; \, 0) = -\frac{5}{4}
    \end{equation*}
    The Residue theorem then gives us:
    \begin{align*}
      I &= -\frac{1}{4i}(2\pi i)(\frac{3}{4} - \frac{5}{4}) \\
      &= -\frac{\pi}{2}(-\frac{1}{2}) \\
      &= \frac{\pi}{4}
    \end{align*}

    \section*{Problem 2}
    The function $f(z)=\frac{z^{2}(z-1)}{\sin^{2}(\pi z)}$ has singularities $\forall z = 0, \pm 1, \, \pm2, \, \pm3, \, \ldots$. To compute the integral using the Residue theorem, we only need to consider the singularities at $z_{0} = 0, \, \pm1$, as the remaining singularities lie outside the contour $|z|=\frac{3}{2}$. For $z_{0} = 0$, we can use the small angle approximation that $\sin(\pi z) \approx \pi z$, $\forall \, |z| \ll 1$. Therefore, as $z\to0$ (i.e. $z$ becomes infinitesimally small):
    \begin{align*}
      \lim_{z\to0}f(z) &\approx \frac{z^{2}(z-1)}{(\pi z)^{2}} \\
      &= \frac{z^{2}(z-1)}{\pi^{2} z^{2}} \\
      &= \frac{(z-1)}{\pi^{2}} \\
      &= -\frac{1}{\pi^{2}} 
    \end{align*}
    As the limit converges, $z_{0}=0$ is a removable singularity, and we therefore have that:
    \begin{equation*}
      \Res(f \, ; \, 0) = 0 
    \end{equation*}
    
    Using the Taylor series of $\sin(z)$ about $z=\pi z_{0}$, we have that:
    \begin{equation*}
      \sin^{2}(\pi z_{0}) = \sin^{2}(\pi z_{0}) + \pi\sin(2\pi z_{0})(z-z_{0}) + \pi^{2}\cos(2\pi z_{0})(z-z_{0})^{2} - 2\pi^{3}\sin(2\pi z_{0})(z-z_{0})^{3} - \ldots
    \end{equation*}

    At $z_{0}=1$, which is a simple pole (simple zero divided by zero of order two), we then have:
    \begin{align*}
      \Res(f \, ; \, 1) &= \lim_{z\to1}(z-1) \frac{z^{2}(z-1)}{\pi^{2}(z-1)^{2} - \ldots} \\
        &= \lim_{z\to1}\cancel{(z-1)} \frac{z^{2}\cancel{(z-1)}}{\cancel{(z-1)^{2}}(\pi^{2} - \frac{8}{3!}\pi^{4}(z-1) + \ldots)} \\
    \end{align*}
    \begin{align*}
      \Res(f \, ; \, 1) &= \lim_{z\to1} \frac{z^{2}}{\pi^{2} + \frac{8}{3!}\pi^{4}(z-1) + \mathcal{O}(z-1)} \\
      &= \frac{1}{\pi^{2}} \\
    \end{align*}
    
    At $z_{0}=-1$, which is a pole of order two. Let the residue at this point be $\mu$. By the Residue theorem, we then have that:
    \begin{align*}
      \oint_{|z|=\frac{3}{2}} f(z) \, dz &= 2\pi i (0 + \frac{1}{\pi^{2}} + \mu) \\
        &= \frac{2i}{\pi} + 2\pi i \mu
    \end{align*}

    
    

    \section*{Problem 3}
    \begin{enumerate}[i)]
      \item The point $z_{0} = 0$ is an essential singularity. The Laurent series of $f(z)$ is:
      \begin{align*}
        f(z) &= z^{3}(\frac{1}{2z} - \frac{1}{3!(2z)^{3}} + \frac{1}{5!(2z)^{5}} - \ldots) \\
          &= \frac{z^{2}}{2} - \frac{z^{3}}{3!(2z)^{3}} + \frac{z^{3}}{5!(2z)^{5}} - \ldots \\
          &= \frac{z^{2}}{2} - \frac{z^{3}}{2^{3}3!z^{3}} + \frac{z^{3}}{2^{5}5!z^{5}} - \ldots \\ 
          &= \frac{z^{2}}{2} - \frac{1}{2^{3}3!} + \frac{1}{2^{5}5!z^{2}} - \ldots \\ 
          &= \frac{1}{2}z^{2} - \frac{1}{2^{3}3!} + \frac{1}{2^{5}5!}z^{-2} - \ldots \\ 
          &= \sum_{n=0}^{\infty}\frac{(-1)^{n}}{2^{2n+1}(2n+1)!}z^{-2(n-1)}
      \end{align*}
      
      
      \item
      \begin{enumerate}[a)]
        \item For the annulus $C(2;0,1)$, we have $0<|z-2|<1$, which is away from the essential singularity, therefore f does not have a singular part in $C(2;0,1)$.
        
        \item 
        For the annulus $C(0;0,\infty)$, the singular part will be the series:
        \begin{align*}
          f(z) &= \sum_{n=2}^{\infty}\frac{(-1)^{n}}{2^{2n+1}(2n+1)!}z^{-2(n-1)} \\
            &= \frac{1}{2^{5}5!}z^{-2} - \frac{1}{2^{7}7!}z^{-4} + \frac{1}{2^{9}9!}z^{-6} - \ldots
        \end{align*}
      \end{enumerate}
      
      \item From the Laurent series above, we observe that all odd-powered terms vanish in the expression, i.e. $a_{-1}=0$. By the Residue theorem, we obtain the result:
      \begin{equation*}
        \oint_{|z|=1} z^{3}\sin(\frac{1}{2z}) \, dz = 0
      \end{equation*}
    \end{enumerate}
    
    
    \section*{Problem 4}
   
    \begin{enumerate}[i)]
      \item
      We can express $f(z)$ as:
      \begin{align*}
        f(z) &= \frac{1}{z^{2}+1} \\
          &= \frac{1}{(z-i)(z+i)} \\
          &= \frac{1}{z-i}\cdot\frac{1}{z+i}
      \end{align*}
     
      
      \item
      \begin{enumerate}[a)]
        \item We have $0<|z-i|<2$. Since $1/(z+i)$ is analytic at $z_{0}=i$:
        \begin{align*}
          \frac{1}{z+i} &= \frac{1}{(z - i) + 2i} \\
            &= \frac{1}{2i} \cdot \frac{1}{(1 + \frac{z-i}{2i})} 
        \end{align*}
        We can verify that $|z-i|/|2i|<1$:
        \begin{gather*}
          \frac{|z-i|}{|2i|} < \frac{2}{2} = 1
        \end{gather*}
  
        We can therefore use the geometric series:
        \begin{align*}
          \frac{1}{z+i} &= \frac{1}{2i} \cdot \frac{1}{(1 - (-\frac{z-i}{2i}))} \\
            &= \frac{1}{2i} \sum_{n=0}^{\infty}(-\frac{z-i}{2i})^{n} \\
          \therefore f(z) &= \frac{1}{z-i}\cdot\frac{1}{z+i} = \frac{1}{z-i}\cdot\frac{1}{2i} \sum_{n=0}^{\infty}(-\frac{z-i}{2i})^{n} \\
          &= \sum_{n=0}^{\infty}(-1)^{n}\frac{(z-i)^{n-1}}{(2i)^{n+1}} \\
        \end{align*}
        
        
        \item We have $|z-i|>2$. Since $1/(z-i)$ is analytic at $z_{0}=-i$:
        \begin{align*}
          \frac{1}{z-i} = \frac{1 + \frac{2i}{z-i}}{z-i} \cdot \frac{1}{1+\frac{2i}{z-i}} 
        \end{align*}

        We can verify that $|2i|/|z-i|<1$:
        \begin{gather*}
          \frac{|z-i|}{|2i|} > \frac{2}{2} = 1 \\
          \Rightarrow \frac{|2i|}{|z-i|} <  1 \\
        \end{gather*}

        We can therefore use the geometric series:
        \begin{align*}
          \frac{1}{z-i} &= \frac{1 + \frac{2i}{z-i}}{z-i} \cdot \frac{1}{1-(-\frac{2i}{z-i})} \\
            &= \frac{1 + \frac{2i}{z-i}}{z-i} \sum_{n=0}^{\infty}(-\frac{2i}{z-i})^{n} \\
            &= (\frac{1}{z-i} + \frac{2i}{(z-i)^{2}}) \sum_{n=0}^{\infty}(-\frac{2i}{z-i})^{n} \\
            &= \sum_{n=0}^{\infty}(\frac{(-1)^{n}(2i)^{n}}{(z-i)^{n+1}} + \frac{(-1)^{n}(2i)^{n+1}}{(z-i)^{n+2}}) \\
          \therefore f(z) &= \frac{1}{z+i}\sum_{n=0}^{\infty}(\frac{(-1)^{n}(2i)^{n}}{(z-i)^{n+1}} + \frac{(-1)^{n}(2i)^{n+1}}{(z-i)^{n+2}})
        \end{align*}
      \end{enumerate}
    \end{enumerate}
    
\end{document}  